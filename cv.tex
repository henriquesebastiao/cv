\documentclass{cv}

\yourname{Henrique Sebastião da Silva Rosa}

\youraddress{
  Desenvolvedor back-end com 1 ano e 8 meses de experiência em projetos escaláveis,
  combinando sólidos conhecimentos em redes de computadores (3+ anos) e sistemas Linux (4+ anos).
  Atuando com Python, focando em arquiteturas RESTful,
  integração contínua e entrega de valor através de soluções cloud-native.
  Além disso, possuo experiência em desenvolvimento para dispositivos embarcados,
  criando firmwares e aplicações para microcontroladores com a plataforma Arduino.
}
\yourportfolio{henriquesebastiao.com/portfolio}
\yourgithub{github.com/henriquesebastiao}
\youremail{contato@henriquesebastiao.com}
\yourwebsite{henriquesebastiao.com}

\begin{document}

\pagestyle{empty}

\maketitle

\section{Habilidades Técnicas}

\subsection{Back-End}

\begin{itemize}
  \item Linguagem: Python
  \item Frameworks e bibliotecas: FastAPI, Django, SQLAlchemy,
    PyTest, Selenium, Cryptography
  \item Bancos de dados: PostgreSQL
\end{itemize}

\subsection{Demais Habilidades}

\begin{itemize}
  \item Ferramentas: Docker, GIT, Linux, Shell Script, OpenTelemetry
  \item Conhecimentos de front-end: HTML, CSS, Bootstrap, TailwindCSS
  \item Desenvolvimento com linguagem C para plataforma Arduino e demais dispositivos embarcados
  \item Experiência com dispositivos Espressif (ESP32, ESP8266)
  \item Gerenciamento de domínios e DNS
\end{itemize}

\section{Projetos}

\project{Manejo}{Modernização do processo de manejo de rebanhos
bovinos}\hfill{Jun 2023 -- momento}\\
\github{henriquesebastiao/modernizacao-manejo-api}\\
\deploy{manejo}
\vspace{0.15cm}

API assíncrona robusta com telemetria e observabilidade para gestão
de rebanhos bovinos construída com FastAPI.

\vspace{0.08cm}

Tecnologias e ferramentas utilizadas nesse projeto:

\begin{itemize}
  \item Desenvolvimento: Python, FastAPI, SQLAlchemy
  \item Implantação: Docker, Nginx
  \item Banco de dados: PostgreSQL
  \item Observabilidade e testes de carga: OpenTelemetry, Prometheus,
    Loki, Tempo, Grafana, Locust
\end{itemize}

\project{Poupy}{Gerência de gastos pessoais}\hfill{Nov 2023 -- Fev 2024}\\
\github{henriquesebastiao/poupy}\\
\deploy{poupy}
\vspace{0.15cm}

Poupy é um aplicativo web para gerenciamento de orçamento e gastos pessoais.

\vspace{0.08cm}

Tecnologias: Python, Django, Docker, PostgreSQL, PyTest, Selenium, HTML e CSS

\project{Saturn}{Firmware para Cardputer com ESP32}\hfill{Jun 2024 -- Jul 2024}\\
\github{henriquesebastiao/saturn}
\vspace{0.15cm}

Saturn é um firmware para ESP32 (mais especificamente o STAMPS3), que implementa diversas funcionalidades
para análises de vulnerabilidades em redes Wi-Fi e dispositivos Bluetooth,
também implementa controle de alguns dispositivos com sensor infravermelho.

\section{Experiência Profissional}

\subsection{Técnico de Redes}
{Jan 2022 -- momento}
\subsubsection{Castilhos Provedor de Internet LTDA}

Como técnico de redes na Castilho@Net tenho sido responsável pelas
seguintes atividades:

\begin{itemize}
  \item Instlação e manutenção de roteadores e sevidores Mikrotik,
    Ubiquiti e Linux.
  \item Configuração de servidores PPPoE, assegurando desempenho e
    estabilidades para os clientes.
  \item Configuração de serviços de rede como VLANs, DHCP e DNS.
  \item Monitoramento da rede com Mikrotik The Dude e Zabbix para
    solução e prevenção de problemas.
  \item Implementação de servidores NAT e Firewalls parasegurança da rede.
\end{itemize}

\section{Educação}

\subsection{Bacharelado em Ciência da Computação}
{Fev 2022 -- momento}
\subsubsection{Universidade Paulista - UNIP}

\end{document}
