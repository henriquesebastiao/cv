\documentclass{cv}

\yourname{Henrique Sebastião da Silva Rosa}

\youraddress{
  Desenvolvedor back-end com 2 anos de experiência em projetos escaláveis,
  combinando sólidos conhecimentos em redes de computadores (3+ anos) e sistemas Linux (4+ anos).
  Atuando com Python, focando em arquiteturas RESTful,
  integração contínua e entrega de valor através de soluções eficientes e robustas.
}

\yourportfolio{henriquesebastiao.com/portfolio}
\yourlinkedin{linkedin.com/in/henriquesebastiao}
\yourgithub{github.com/henriquesebastiao}
\youremail{contato@henriquesebastiao.com}
\yourwebsite{henriquesebastiao.com}

\begin{document}

\pagestyle{empty}

\maketitle

\section{Habilidades Técnicas}

\vspace{0.15cm}

\begin{itemize}
  \item Linguagem: Python
  \item Principais frameworks e bibliotecas: FastAPI, Django, SQLAlchemy, PyTest e Selenium
  \item Ferramentas: Docker, GIT, PostgreSQL, Linux, Shell Script, OpenTelemetry, Zabbix, Grafana e Cloudflare
  \item Conhecimentos de front-end: HTML, CSS, Bootstrap e TailwindCSS
  \item Microsoft Office e Google Workspace
\end{itemize}

\section{Projetos}

\project{Dotum}{Sistema de contas a pagar e a receber}\hfill{Jul 2025}\\
\github{henriquesebastiao/dotum}\\
\deploy{dotum-web}
\vspace{0.15cm}

Sistema de contas a pagar e a receber com back-end em FastAPI e interface web em Streamlit.
Permite registrar e acompanhar contas com dados como valor, descrição, vencimento, tipo
(pagar ou receber) e status de pagamento. Exibe resumo com totais, balanço geral e gráfico das finanças.

\vspace{0.08cm}

Tecnologias: Python, FastAPI, Streamlit, SQLAlchemy, Docker e PostgreSQL.

\project{Poupy}{Gerência de gastos pessoais}\hfill{Nov 2023 -- Fev 2024}\\
\github{henriquesebastiao/poupy}\\
\deploy{poupy}
\vspace{0.15cm}

Poupy é um app web de controle financeiro pessoal desenvolvido com Django.
Permite gerenciar contas, receitas, despesas e transferências, oferecendo
um dashboard com resumo mensal para organização das finanças.

\vspace{0.08cm}

Tecnologias: Python, Django, Docker, PostgreSQL, PyTest, Selenium, HTML e CSS.

\project{Manejo}{Modernização do processo de manejo de rebanhos
bovinos}\hfill{Jun 2023 -- momento}\\
\github{henriquesebastiao/modernizacao-manejo-api}\\
\deploy{manejo}
\vspace{0.15cm}

Projeto de modernização do manejo focado no controle eficiente da produção e dos
gastos com bovinos, oferecendo suporte à tomada de decisões por meio da análise de performance do rebanho.

\vspace{0.08cm}

Tecnologias: FastAPI, SQLAlchemy, Docker, Nginx, PostgreSQL, OpenTelemetry, Prometheus e Grafana.

\section{Experiência Profissional}

\subsection{Técnico de Redes}
{Jan 2022 -- momento}
\subsubsection{Castilhos Provedor de Internet LTDA}

Como técnico de redes na Castilho@Net tenho sido responsável pelas
seguintes atividades:

\begin{itemize}
  \item Instalação e manutenção de roteadores e servidores Mikrotik,
    Ubiquiti e Linux.
  \item Configuração de servidores PPPoE, assegurando desempenho e
    estabilidades para os clientes.
  \item Configuração de serviços de rede como VLANs, DHCP e DNS.
  \item Monitoramento da rede com Mikrotik The Dude e Zabbix para
    solução e prevenção de problemas.
  \item Implementação de servidores NAT e Firewalls para segurança da rede.
\end{itemize}

\section{Educação}

\subsection{Bacharelado em Ciência da Computação}
{Fev 2022 -- Dez 2025}
\subsubsection{Universidade Paulista - UNIP}

\end{document}
