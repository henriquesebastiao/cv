\documentclass{cv}

\yourname{Henrique Sebastião da Silva Rosa}

\youraddress{
  Técnico de redes com mais de 3 anos de experiência atuando em provedores de internet, com foco na configuração, manutenção e monitoramento de redes com equipamentos Mikrotik, Ubiquiti e Intelbras, além de domínio em protocolos PPPoE, L2TP, VLANs e roteamento estático/dinâmico. Possuo conhecimento avançado em sistemas Linux (4+ anos), automação de tarefas com Bash e Python, e no uso de ferramentas de monitoramento como Zabbix. Tenho vivência em atendimento a clientes, suporte de 1º e 2º nível, participando ativamente da operação e expansão da infraestrutura de rede, prezando pela estabilidade dos serviços e pela resolução eficiente de incidentes. Busco evoluir para cargos de liderança, aplicando minha bagagem técnica para contribuir estrategicamente com o crescimento do provedor.
}
\youremail{contato@henriquesebastiao.com}
\yourwebsite{henriquesebastiao.com}

\begin{document}

\pagestyle{empty}

\maketitle

\section{Habilidades Técnicas}

\subsection{Gerenciamento de redes}

\begin{itemize}
  \item Administração de roteadores Mikrotik, Ubiquiti, entre outros.
  \item Gerenciamento de domínios e DNS.
  \item Implementação de protocolos de VPN como PPPoE, L2TP e Wireguard.
  \item Configuração de ferramentas de monitoramento com dashboards via SNMP como Zabbix e Grafana.
  \item Automação de tarefas em roteadores Mikrotik com \textit{scripts} RouterOS.
\end{itemize}

\subsection{Demais habilidades e conhecimentos}

\begin{itemize}
  \item Manutenção de torres de transmissão de internet via rádio.
  \item Suporte técnico a clientes de internet via rádio e fibra óptica.
  \item Instalação e manutenção de câmeras de segurança CFTV.
  \item Ferramentas: Docker, GIT e Linux.
  \item Experiência com plataformas Arduino e Espressif (ESP32, ESP8266).
  \item Linguagem de programação e \textit{script}: Python e Shell Script.
\end{itemize}

\section{Experiência Profissional}

\subsection{Técnico de Redes}
{Jan 2022 -- momento}
\subsubsection{Castilhos Provedor de Internet LTDA}

Como técnico de redes na Castilho@Net tenho sido responsável pelas
seguintes atividades:

\begin{itemize}
  \item Instalação e manutenção de roteadores e sevidores Mikrotik,
    Ubiquiti e Linux.
  \item Configuração de servidores PPPoE, assegurando desempenho e
    estabilidades para os clientes.
  \item Configuração de serviços de rede como VLANs, DHCP e DNS.
  \item Monitoramento da rede com Mikrotik The Dude e Zabbix para
    solução e prevenção de problemas.
  \item Implementação de servidores NAT e Firewalls parasegurança da rede.
\end{itemize}

\section{Educação}

\subsection{Bacharelado em Ciência da Computação}
{Fev 2022 -- momento}
\subsubsection{Universidade Paulista - UNIP}

\section{Certificações}

\subsection{Cursos}

\begin{itemize}
  \item Redes onboarding: uma perspectiva prática - Alura (2022).
  \item DNS: entenda a resolução de nomes na internet - Alura (2022).
  \item Linux Onboarding: usando a CLI de uma forma rápida e prática - Alura (2022).
  \item Linux Onboarding: localizando arquivos e conteúdos - Alura (2022).
  \item Linux I: conhecendo e utilizando o terminal - Alura (2022).
  \item  Linux II: programas, processos e pacotes - Alura (2022).
\end{itemize}

\end{document}
